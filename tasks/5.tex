Пусть мы знаем, что для любых двух точек $x_1$ и $x_2$ и выпуклой вниз функции $f(x)$, а также любых двух чисел $q_1$ и $q_2 \geq 0$, таких что $q_1+q_2=1$ выполнено $q_1f(x_1)+q_2f(x_2) \geq f(q_1x_1+ q_2x_2)$ (определение выпуклости). Предположим, что для любых $x_1, x_2, ..., x_n$, выпуклой вверх функции $f(x)$ и любых $q_1, q_2, ..., q_n \geq 0$, т. ч. $\sum_{i=1}^n q_i = 1$ выполнено $\sum_{i=1}^n q_if(x_i) \geq f(\sum_{i=1}^n q_ix_i)$. Докажем, что и для любых $x_1, x_2, ..., x_{n+1}$, этой же функции $f(x)$ и любых $q_1, q_2, ..., q_{n+1} \geq 0$, т. ч. $\sum_{i=1}^{n+1} q_i = 1$ выполнено $\sum_{i=1}^{n+1} q_if(x_i) \geq f(\sum_{i=1}^{n+1} q_ix_i)$. Из предположения следует $(q_1+q_2)f(\frac{q_1x_1 + q_2x_2}{q_1+q_2}) + ... + q_{n+1}f(x_{n+1}) \geq f((q_1+q_2)\frac{q_1x_1 + q_2x_2}{q_1+q_2} + ...  + q_{n+1}x_{n+1})$. Сравним левую часть этого неравенства с $\sum_{i=1}^{n+1} q_if(x_i)$, удалив все повторяющиеся члены получим, что должно выполняться: $q_1f(x_1)+q_2f(x_2) \geq (q_1+q_2)f(\frac{q_1x_1 + q_2x_2}{q_1+q_2})$ Поделим на $q_1+q_2$, получим $\frac{q_1}{q_1+q_2}f(x_1)+\frac{q_2}{q_1+q_2}f(x_2) \geq f(x_1\frac{q_1}{q_1+q_2} + x_2\frac{q_2}{q_1+q_2})$, что правда по определению выпуклости, т. к. $\frac{q_1}{q_1+q_2} + \frac{q_2}{q_1+q_2} = 1$.
