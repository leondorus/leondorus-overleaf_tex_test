Составим двудольный граф таким образом, что левая доля это столбы а правая это строки, ребро между вершинами есть если на пересечении этого столбца и этой строки стоит положительное число. По теореме Кёнига, мы знаем, что наибольшее паросочетание (то что нас просят найти) равно наименьшему вершинному покрытию. Докажем что последнее равно $n$. Понятно что за $n$ мы справимся (просто выделим все вершины в одной доле), докажем что за меньшее количество нельзя. Пусть мы составили вершинное покрытие из $k<n$ вершин. Тогда сумма чисел на ребрах покрытых этими вершинами $\leq k = k \cdot 1$, т. к. если ребро покрыто два раза, то оно учитывается только один раз, а сумма чисел под ребрами для каждой вершины равна $1$. Но с другой стороны сумма чисел во всей таблице равна $n > k$, противоречие. Следовательно наименьшее вершинное покрытие равно $n$ равно наибольшему паросочетанию ч.т.д.